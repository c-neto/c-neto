\newcommand{\RNum}[1]{\uppercase\expandafter{\romannumeral #1\relax}}
%% If you need to pass whatever options to xcolor
\PassOptionsToPackage{dvipsnames}{xcolor}

%% If you are using \orcid or academicons
%% icons, make sure you have the academicons 
%% option here, and compile with XeLaTeX
%% or LuaLaTeX.
% \documentclass[10pt,a4paper,academicons]{altacv}

%% Use the "normalphoto" option if you want a normal photo instead of cropped to a circle
% \documentclass[10pt,a4paper,normalphoto]{altacv}

\documentclass[10pt,a4paper]{altacv}
%% AltaCV uses the fontawesome and academicon fonts
%% and packages. 
%% See texdoc.net/pkg/fontawecome and http://texdoc.net/pkg/academicons for full list of symbols.
%% 
%% Compile with LuaLaTeX for best results. If you
%% want to use XeLaTeX, you may need to install
%% Academicons.ttf in your operating system's font 
%% folder.


% Change the page layout if you need to
\geometry{left=1cm,right=9cm,marginparwidth=6.8cm,marginparsep=1.2cm,top=1.25cm,bottom=1.25cm,footskip=2\baselineskip}

% Change the font if you want to.

% If using pdflatex:
\usepackage[T1]{fontenc}
\usepackage[utf8]{inputenc}
\usepackage[default]{lato}

% If using xelatex or lualatex:
% \setmainfont{Lato}

% Change the colours if you want to
\definecolor{Navy}{HTML}{000080}
\definecolor{SlateGrey}{HTML}{2E2E2E}
\definecolor{LightGrey}{HTML}{444444}
\colorlet{heading}{Navy}
\colorlet{accent}{Navy}
\colorlet{emphasis}{SlateGrey}
\colorlet{body}{LightGrey}

% Change the bullets for itemize and rating marker
% for \cvskill if you want to
\renewcommand{\itemmarker}{{\small\textbullet}}
\renewcommand{\ratingmarker}{\faCircle}
%% sample.bib contains your publications
\addbibresource{sample.bib}

\usepackage[colorlinks]{hyperref}

\begin{document}

\name{Carlos Augusto dos Santos Neto}
\tagline{DEVOPS | DESENVOLVEDOR INTEGRADOR}

\personalinfo{
    \mailaddress{
        \href{mailto:carlos.neto.dev@gmail.com}
        {carlos.neto.dev@gmail.com}
    }
    \phone{
        \href{tel:5512987078145}
        % {+55 12 987078145}
        {12987078145}
    }
    \linkedin{
        \href{https://www.linkedin.com/in/c-neto/}
        {c-neto}
    }
    \github{
        \href{https://github.com/augustoliks}
        {augustoliks} 
    }
    \location{
        \href{https://maps.google.com/maps?q=-23.296193,-46.027498}
        {Jacareí-SP}
    }
    \age{
        \href{https://www.google.com/search?q=data+de+nascimento+23-09-1997}
        {23-09-1997}
    }
    \martialstatus{
        \href{https://www.google.com/search?q=Estado+de+Civil+Solteiro}
        {solteiro}
    }
}

%% Make the header extend all the way to the right, if you want. 
\begin{fullwidth}
    \makecvheader
\end{fullwidth}

%% Depending on your tastes, you may want to make fonts of itemize environments slightly smaller
% \AtBeginEnvironment{itemize}{\small}


%% Provide the file name containing the sidebar contents as an optional parameter to \cvsection.
%% You can always just use \marginpar{...} if you do
%% not need to align the top of the contents to any
%% \cvsection title in the "main" bar.
\cvsection[page1sidebar]{Experiência}

\cvevent{Programador Júnior 2}{Fotosensores Tecnologia Eletrônica LTDA | Mobilidade Urbana}{Agosto 2020 -- Presente}{São José dos Campos-SP}
\begin{itemize}
    \item Desenvolvimento de bibliotecas, scripts, sistemas e integrações com Python;
    \item Desenvolvimento de utilitários de linha de comando com Go;
    % \item Preparação de ambientes de produção com Docker Compose, criação e otimização de imagens Docker e provisionamento com Ansible;
    \item Provisionamento e customização de sistemas com Ansible;
    % \item Integração de sistemas através de API com FastAPI e Flask;
    \item Ingestão de logs com Promtail, Rsyslog, Fluent Bit e Journald;
    \item Criação de plugins Rsyslog com Python e Go;
    \item Coleta e padronização de logs com Loki e Graylog; 
    \item Coleta e  padronização de métricas com Prometheus e Zabbix;
    \item Criação de dashboards de observabilidade de sistemas com Grafana;
    % \item Criação e padronização de rotinas CI/CD com o Gitlab;
    \item Documentação com MKDocs e Sphinx;
    \item Configuração de Load Balancer e Reverse Proxy com Traefik;
    \item Configuração de servidores Web com NGINX (SSL/TLS);
    \item Configuração de repositórios Yum, PyPI e Docker Registry com Nexus Repository;
    \item Padronização de rotinas de CI/CD com Gitlab integrado ao Nexus Repository para a construção de:
        \begin{itemize}
            \item Pacotes RPM, construídos com rpm-build;
            \item Imagens de Containers, construídas com Docker e Buildah;
            \item Bibliotecas Python, construídas com Poetry e pip;
            \item Bibliotecas Java, construídas com Maven.
        \end{itemize}
    % \item Estudos técnicos com base nas regras de negócio da empresa;
    % (\href{https://pt.wikipedia.org/wiki/Prova_de_conceito}{PoC})

\end{itemize}

\divider

\cvevent{Programador Júnior 1}{Fotosensores Tecnologia Eletrônica LTDA | Mobilidade Urbana}{Junho 2019 -- Agosto 2020}{São José dos Campos-SP}
\begin{itemize}
    \item Automação de testes de performance de sensores metrológicos de trânsito com Python;
    \item Automação de tarefas de infra com Python e Shell Script.
\end{itemize}

\cvsection{Projetos}

\cvevent{Sistema de Monitoramento de Equipamentos Fiscalizadores de Trânsito - Fotosensores LTDA.}
{\MakeLowercase{python, bash, ansible, fastapi, zabbix, grafana, nginx, LDAP, gitlab ci/cd}} {}{}
\begin{itemize}
    \item Apresenta dados de monitoramento de forma intuitiva e objetiva. Contribui diretamente na constatação de problemas nos equipamentos, agilizando o processo de abertura de Ordem de Serviço de manutenção técnica.
\end{itemize}
\divider

% \cvevent{Plugin de Inventário Ansible}
% {\MakeLowercase{ansible, python, fastapi}}{}{}
% \begin{itemize}
%     % \item Plugin de inventário Ansible, criada para obter equipamentos de forma costumizável, por meio de uma API HTTP;
%     \item \github{\href{https://github.com/augustoliks/ansible-ws-inventory-plugin}{https://github.com/augustoliks/ansible-ws-inventory-plugin}}
% \end{itemize}
% \divider

% \cvevent{Integração de Autenticação NGINX com LDAP}
% {\MakeLowercase{python, flask, nginx, ldap, docker}}{}{}
% \begin{itemize}
%     % \item Autenticação em um servidor NGINX, utilizado a funcionaliade via sub requisição;
%     \item \github{\href{https://github.com/augustoliks/nginx-auth-proxy-ldap}{https://github.com/augustoliks/nginx-auth-proxy-ldap}}
% \end{itemize}

\cvevent{Sistema de Transmissão de Dados - Fotosensores LTDA.}
{\MakeLowercase{python, rpm, nginx, traefik, docker}}{}{}
\begin{itemize}
    \item Responsável pelo envio de dados dos equipamentos fiscalizadores de trânsito. Focado em confiabilidade e integridade, o sistema evita inconstência de dados, contribuindo com o SLA dos contratos.
\end{itemize}


\clearpage

%% If the NEXT page doesn't start with a \cvsection but you'd
%% still like to add a sidebar, then use this command on THIS
%% page to add it. The optional argument lets you pull up the 
%% sidebar a bit so that it looks aligned with the top of the
%% main column.
% \addnextpagesidebar[-1ex]{page3sidebar}

\end{document}
