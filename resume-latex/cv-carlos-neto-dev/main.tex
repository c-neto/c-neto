% Depending on your tastes, you may want to make fonts of itemize environments slightly smaller
% \AtBeginEnvironment{itemize}{\small}


%% Provide the file name containing the sidebar contents as an optional parameter to \cvsection.
%% You can always just use \marginpar{...} if you do
%% not need to align the top of the contents to any
%% \cvsection title in the "main" bar.

\newcommand{\RNum}[1]{\uppercase\expandafter{\romannumeral #1\relax}}
%% If you need to pass whatever options to xcolor
\PassOptionsToPackage{dvipsnames}{xcolor}

%% If you are using \orcid or academicons
%% icons, make sure you have the academicons 
%% option here, and compile with XeLaTeX
%% or LuaLaTeX.
% \documentclass[10pt,a4paper,academicons]{altacv}

%% Use the "normalphoto" option if you want a normal photo instead of cropped to a circle
% \documentclass[10pt,a4paper,normalphoto]{altacv}

\documentclass[10pt,a4paper]{altacv}
%% AltaCV uses the fontawesome and academicon fonts
%% and packages. 
%% See texdoc.net/pkg/fontawecome and http://texdoc.net/pkg/academicons for full list of symbols.
%% 
%% Compile with LuaLaTeX for best results. If you
%% want to use XeLaTeX, you may need to install
%% Academicons.ttf in your operating system's font 
%% folder.


% Change the page layout if you need to
\geometry{left=1cm,right=9cm,marginparwidth=6.8cm,marginparsep=1.2cm,top=1.25cm,bottom=1.25cm,footskip=2\baselineskip}

% Change the font if you want to.

% If using pdflatex:
\usepackage[T1]{fontenc}
\usepackage[utf8]{inputenc}
\usepackage[default]{lato}


\setlength{\bigskipamount}{4pt} 

% If using xelatex or lualatex:
% \setmainfont{Lato}

% Change the colours if you want to
\definecolor{Navy}{HTML}{000080}
\definecolor{SlateGrey}{HTML}{2E2E2E}
\definecolor{LightGrey}{HTML}{444444}
\colorlet{heading}{Navy}
\colorlet{accent}{Navy}
\colorlet{emphasis}{SlateGrey}
\colorlet{body}{LightGrey}

% Change the bullets for itemize and rating marker
% for \cvskill if you want to
\renewcommand{\itemmarker}{{\small\textbullet}}
\renewcommand{\ratingmarker}{\faCircle}
%% sample.bib contains your publications
\addbibresource{sample.bib}

\begin{document}

\name{Carlos Neto}

\tagline{DEVOPS ENGINEER}

\personalinfo{
    \homepage{
        \href{https://www.carlosneto.dev/}
        {\uline{www.carlosneto.dev}}
    }
    \linkedin{
        \href{https://www.linkedin.com/in/c-neto/}
        {\uline{linkedin.com/in/c-neto/}}
    }
    \github{
        \href{https://github.com/c-neto}
        {\uline{github.com/c-neto}}
    }
    \age{
        \href{https://www.credly.com/users/c-neto/badges}
        {\uline{credly.com/c-neto/badges}}
    }
    \location{
        \href{https://en.wikipedia.org/wiki/Sao_Paulo}
        {\uline{Brazil - São Paulo}}
    }
}

%% Make the header extend all the way to the right, if you want. 
\begin{fullwidth}
    \makecvheader
\end{fullwidth}

\bigskip


\cvsection[page1sidebar]{PROFESSIONAL PROFILE}

Self-motivated professional with a strong sense of \textbf{accountability} and \textbf{ownership}, recognized for \textbf{high performance} and \textbf{documentation} skills.

\bigskip

% Experienced in high availability and fault-tolerant systems, particularly within Open Banking platforms and IoT environments. 

DevOps Engineer with \textbf{programming} background and expertise in observability, focusing on log analytics. Specializes in Python, Fluentbit, Logstash, OpenSearch, AWS, and Kubernetes workload management.

\bigskip

Technical writer with \href{https://carlosneto.dev/whoami/#opensource-contributions}{\uline{OpenSource contributions}} in the official documentation for OpenSearch, External Secrets Operator, and OpenCTI. Maintains the \href{https://www.carlosneto.dev/}{\uline{www.carlosneto.dev}} blog, sharing insights and solutions on DevOps practices.


\bigskip
\bigskip

\cvsection[page1sidebar]{Professional Experience}

\cvevent{DevOps Engineer}{\uline{\href{https://us.trustly.com/}{Trustly | Open Banking Platform} } }{November 2021 -- Current}{Remote}

Works on a global Account-to-Account Open Banking system running in the AWS Cloud. Primary tasks:

\bigskip

\begin{itemize}
    \item Managing a large self-hosted OpenSearch cluster, including performance tuning, SSO integration, RBAC management, ISM policies, monitoring, template creation, and cost optimization.
    \item Creating log analytics pipelines (Fluentbit + Logstash + OpenSearch).
    \item Developing automation scripts and CLI (Python + Bash).
    \item Managing Kubernetes workloads and creating Helm Charts.
    \item Optimizing AWS resources with a focus on performance and security.
    \item Developing Scrum reports (Jira API + Pandas +  Google Sheets).
    \item Writing and maintaining Knowledge Base documentation.
\end{itemize}

\divider

\cvevent{Software Developer}{\uline{\href{http://fotosensores.com/}{Fotosensores | Speed Camera Systems}}}{April, 2018 -- October, 2021}{On-site}

Worked on a regional Speed Camera IoT system running on self-hosted infrastructure. Primary tasks:

\bigskip

\begin{itemize}
    \item Developing Python microservices, scripts, CLI tools, and Web APIs.
    \item Implementing observability (Zabbix + Prometheus + Grafana).
    \item Creating log analytics pipelines (Rsyslog + Loki + Grafana).
    \item Provisioning Linux server setups (Ansible + AWX).  
    \item Packaging Python, Java, Go application in RPM format.
    \item Writing and maintaining Knowledge Base documentation.  
\end{itemize}

\bigskip
\bigskip

\cvsection[page1sidebar]{Projects}

\begin{itemize}
    \item \textbf{OpenSearch Tuning (Trustly)}: Implemented RBAC security practices, observability, and optimized index templates and ISM policies.
    \item \textbf{Log Analytics (Trustly)}: Built log ingestion pipelines for applications running on Kubernetes (Fluentbit + Logstash + OpenSearch).
    \item \textbf{SFTP Cluster (Trustly)}: Developed a highly available SFTP server with fault tolerance across multiple AZs (Kubernetes + Ansible).
    \item \textbf{Speed Camera Monitoring (Fotosensores)}: Developed a web system for monitoring speed camera components (Grafana + Zabbix + Flask).
\end{itemize}


\clearpage
\end{document}
