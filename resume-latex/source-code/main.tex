% Depending on your tastes, you may want to make fonts of itemize environments slightly smaller
% \AtBeginEnvironment{itemize}{\small}


%% Provide the file name containing the sidebar contents as an optional parameter to \cvsection.
%% You can always just use \marginpar{...} if you do
%% not need to align the top of the contents to any
%% \cvsection title in the "main" bar.

\newcommand{\RNum}[1]{\uppercase\expandafter{\romannumeral #1\relax}}
%% If you need to pass whatever options to xcolor
\PassOptionsToPackage{dvipsnames}{xcolor}

%% If you are using \orcid or academicons
%% icons, make sure you have the academicons 
%% option here, and compile with XeLaTeX
%% or LuaLaTeX.
% \documentclass[10pt,a4paper,academicons]{altacv}

%% Use the "normalphoto" option if you want a normal photo instead of cropped to a circle
% \documentclass[10pt,a4paper,normalphoto]{altacv}

\documentclass[10pt,a4paper]{altacv}
%% AltaCV uses the fontawesome and academicon fonts
%% and packages. 
%% See texdoc.net/pkg/fontawecome and http://texdoc.net/pkg/academicons for full list of symbols.
%% 
%% Compile with LuaLaTeX for best results. If you
%% want to use XeLaTeX, you may need to install
%% Academicons.ttf in your operating system's font 
%% folder.


% Change the page layout if you need to
\geometry{left=1cm,right=9cm,marginparwidth=6.8cm,marginparsep=1.2cm,top=1.25cm,bottom=1.25cm,footskip=2\baselineskip}

% Change the font if you want to.

% If using pdflatex:
\usepackage[T1]{fontenc}
\usepackage[utf8]{inputenc}
\usepackage[default]{lato}

% If using xelatex or lualatex:
% \setmainfont{Lato}

% Change the colours if you want to
\definecolor{Navy}{HTML}{000080}
\definecolor{SlateGrey}{HTML}{2E2E2E}
\definecolor{LightGrey}{HTML}{444444}
\colorlet{heading}{Navy}
\colorlet{accent}{Navy}
\colorlet{emphasis}{SlateGrey}
\colorlet{body}{LightGrey}

% Change the bullets for itemize and rating marker
% for \cvskill if you want to
\renewcommand{\itemmarker}{{\small\textbullet}}
\renewcommand{\ratingmarker}{\faCircle}
%% sample.bib contains your publications
\addbibresource{sample.bib}

\usepackage[colorlinks]{hyperref}

\begin{document}

\name{Carlos Neto}

\tagline{DEVOPS ENGINEER}

\personalinfo{
    % \mailaddress{
    %     \href{mailto:x.dev@gmail.com}
    %     {x@gmail.com}
    % }
    \linkedin{
        \href{https://www.linkedin.com/in/c-neto/}
        {https://www.linkedin.com/in/c-neto/}
    }
    \github{
        \href{https://github.com/augustoliks}
        {https://github.com/augustoliks} 
    }
    \location{
        \href{https://en.wikipedia.org/wiki/Sao_Paulo}
        {Brazil - São Paulo}
    }
}

%% Make the header extend all the way to the right, if you want. 
\begin{fullwidth}
    \makecvheader
\end{fullwidth}


\cvsection[page1sidebar]{Personal Summary}

Self-learning and Ownership mindset IT professional, with tech skills in Programming, Documentation, and DevOps.

\bigskip

It is a Python programming specialist, experienced in Amazon Web Services, focused on system provisioning, automation scripts, and managing logs pipelines and observability stacks.

\cvsection[page1sidebar]{Employment History}

\cvevent{DevOps Engineer}{Trustly Inc. | Online Banking Payments}{November 2021 -- Current}{Remote}

DevOps Engineer focused on Cloud programming tasks and log pipeline development. Responsible for automation and Cloud components configuration in a world system-wide online banking platform.

\bigskip

Main tasks carried out:

\bigskip

\begin{itemize}
    \item Automation scripts with Python and Bash;
    \item OpenSearch cluster management (RBAC, ISM, templates);
    \item Logstash log pipelines (Enriching, Grok);
    \item AWS resources analysis and configuration;
    \item Kubernetes operation, troubleshooting, and development;
    \item Knowledge Base documentation.
\end{itemize}    

\divider

\cvevent{Software Developer}{Fotosensores | Speed Camera Systems}{April, 2018 -- October, 2021}{On-site}

Software Developer focused on In-House infrastructure demands. Responsible for microservices development, system provisioning, and observability in a regional system-wide Speed Camera System.

\bigskip

Main tasks carried out:

\bigskip

\begin{itemize}
    \item Python microservices, scripts, and integrations;
    \item GitLab CI/CD pipelines;
    \item Observability and monitoring with Zabbix, Prometheus, and Grafana;
    \item Linux Servers management and packing (Red Hat based);
    \item Knowledge Base documentation.
\end{itemize}

\cvsection[page1sidebar]{Projects}

\begin{itemize}
    \item \textbf{SFTP Cluster (Trustly Inc.)}: High available SFTP server with users setup provisioned by Ansible and deployment over Kubernetes;
    \item \textbf{OpenSearch RBAC (Trustly Inc.)}: RBAC roles provisioned by OpenSearch Security Plugin based on Principle of Least Privilege;
    \item \textbf{Image Transmissions Module (Fotosensores)}: Speed camera systems module to send vehicle images with integrity and cryptography built;
    \item \textbf{Speed Camera Monitoring System (Fotosensores)}: Web system focused on speed camera system components observability.
\end{itemize}




\clearpage
\end{document}
