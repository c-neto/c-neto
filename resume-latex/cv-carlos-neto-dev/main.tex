% Depending on your tastes, you may want to make fonts of itemize environments slightly smaller
% \AtBeginEnvironment{itemize}{\small}


%% Provide the file name containing the sidebar contents as an optional parameter to \cvsection.
%% You can always just use \marginpar{...} if you do
%% not need to align the top of the contents to any
%% \cvsection title in the "main" bar.

\newcommand{\RNum}[1]{\uppercase\expandafter{\romannumeral #1\relax}}
%% If you need to pass whatever options to xcolor
\PassOptionsToPackage{dvipsnames}{xcolor}

%% If you are using \orcid or academicons
%% icons, make sure you have the academicons 
%% option here, and compile with XeLaTeX
%% or LuaLaTeX.
% \documentclass[10pt,a4paper,academicons]{altacv}

%% Use the "normalphoto" option if you want a normal photo instead of cropped to a circle
% \documentclass[10pt,a4paper,normalphoto]{altacv}

\documentclass[10pt,a4paper]{altacv}
%% AltaCV uses the fontawesome and academicon fonts
%% and packages. 
%% See texdoc.net/pkg/fontawecome and http://texdoc.net/pkg/academicons for full list of symbols.
%% 
%% Compile with LuaLaTeX for best results. If you
%% want to use XeLaTeX, you may need to install
%% Academicons.ttf in your operating system's font 
%% folder.


% Change the page layout if you need to
\geometry{left=1cm,right=9cm,marginparwidth=6.8cm,marginparsep=1.2cm,top=1.25cm,bottom=1.25cm,footskip=2\baselineskip}

% Change the font if you want to.

% If using pdflatex:
\usepackage[T1]{fontenc}
\usepackage[utf8]{inputenc}
\usepackage[default]{lato}

% If using xelatex or lualatex:
% \setmainfont{Lato}

% Change the colours if you want to
\definecolor{Navy}{HTML}{000080}
\definecolor{SlateGrey}{HTML}{2E2E2E}
\definecolor{LightGrey}{HTML}{444444}
\colorlet{heading}{Navy}
\colorlet{accent}{Navy}
\colorlet{emphasis}{SlateGrey}
\colorlet{body}{LightGrey}

% Change the bullets for itemize and rating marker
% for \cvskill if you want to
\renewcommand{\itemmarker}{{\small\textbullet}}
\renewcommand{\ratingmarker}{\faCircle}
%% sample.bib contains your publications
\addbibresource{sample.bib}

\begin{document}

\name{Carlos Neto}

\tagline{DEVOPS ENGINEER}

\personalinfo{
    \homepage{
        \href{https://www.carlosneto.dev/}
        {\uline{www.carlosneto.dev}}
    }
    \linkedin{
        \href{https://www.linkedin.com/in/c-neto/}
        {\uline{linkedin.com/in/c-neto/}}
    }
    \github{
        \href{https://github.com/c-neto}
        {\uline{github.com/c-neto}}
    }
    \age{
        \href{https://www.credly.com/users/c-neto/badges}
        {\uline{credly.com/c-neto/badges}}
    }
    \location{
        \href{https://en.wikipedia.org/wiki/Sao_Paulo}
        {\uline{Brazil - São Paulo}}
    }
}

%% Make the header extend all the way to the right, if you want. 
\begin{fullwidth}
    \makecvheader
\end{fullwidth}


\cvsection[page1sidebar]{Summary}

Dedicated and self-motivated IT professional with a strong commitment to continuous learning and a deep sense of ownership. 

\bigskip

Proficient in a range of technical skills including programming, documentation, and DevOps. Specialized in Python programming and well-versed in Amazon Web Services, with a primary focus on system provisioning, creating automation scripts, and managing log pipelines and observability stacks.

\cvsection[page1sidebar]{Professional Experience}

\cvevent{DevOps Engineer}{\uline{\href{https://us.trustly.com/}{Trustly | Open Banking Platform} } }{November 2021 -- Current}{Remote}

I work on a global Account-to-Account Open Banking System, where my primary responsibilities encompass scripting, cloud component optimization, and log pipeline creation. My core tasks include:

\bigskip

\begin{itemize}
    \item Crafting automation scripts using Python and Bash.
    \item Managing a massive OpenSearch cluster, including Cluster configuration, RBAC, and Template tuning.
    \item Developing and maintaining Logstash log pipelines.
    \item Optimizing AWS resources focuses on performance and security.
    \item Kubernetes operating, troubleshooting, and optimization.
    \item Knowledge Base documentation.
\end{itemize}


\divider

\cvevent{Software Developer}{\uline{\href{http://fotosensores.com/}{Fotosensores | Speed Camera Systems}}}{April, 2018 -- October, 2021}{On-site}

I focus on meeting in-house infrastructure needs, primarily within a regional Speed Camera System. My key responsibilities include:

\bigskip

\begin{itemize}
    \item Developing Python microservices, scripts, and integrations.
    \item Managing GitLab CI/CD pipelines.
    \item Setting up monitoring with Zabbix, Prometheus, and Grafana.
    \item Developing Log Pipelines with Rsyslog, Loki, and Grafana.
    \item Handling Linux server and RPM packaging (CentOs-based).
    \item Maintaining knowledge base documentation.
\end{itemize}

\cvsection[page1sidebar]{Projects}

\begin{itemize}
    \item \textbf{SFTP Cluster (Trustly)}: Implemented a highly available SFTP server using Ansible for user setup and Kubernetes for deployment.
    \item \textbf{OpenSearch Security and Monitoring (Trustly)}: Implemented robust monitoring (Prometheus + Grafana), and security best practices, including the Principle of Least Privilege, and Defense-in-Depth.
    \item \textbf{Image Transmissions Module (Fotosensores)}: Developed a module for speed camera systems that securely transmits vehicle images with data integrity and cryptography.
    \item \textbf{Speed Camera Monitoring System (Fotosensores)}: Created a web-based system for monitoring and observing components of the speed camera system.
\end{itemize}

\clearpage
\end{document}
